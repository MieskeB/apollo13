\section{Gezondheid expert}

Welkom in de rol van de Gezondheid Expert. Jouw primaire missiedoel is ervoor zorgen dat de astronauten heelhuids thuis komen. Hierbij moet je verschillende biometrie in de gaten houden en aansluitend advies geven aan het team hierover. Deze uitgebreide handleiding zal je voorzien van de kennis en procedures die nodig zijn om je essentiële rol te vervullen.

\subsection{Overzicht van vitaliteiten}

\subsubsection{Hartslag}
Ieder levend persoon moet een hartslag (heart beat rate) hebben. Dit is essentieel omdat de hartslag verantwoordelijk is voor het rondpompen van bloed door het lichaam. Het bloed transporteert zuurstof en voedingsstoffen naar de cellen en voert afvalstoffen, zoals koolstofdioxide, af. Zonder een constante en regelmatige hartslag zou het lichaam niet in staat zijn om zijn functies te onderhouden. Een afwijkende hartslag kan een teken zijn van stress, ziekte, of problemen met de hartfunctie, wat de algehele gezondheid aanzienlijk kan beïnvloeden. Daarom wordt de hartslag vaak gemonitord om het cardiovasculaire systeem in de gaten te houden.

\subsubsection{Bloeddruk}
Een bloeddruk (blood pressure) is belangrijk om te hebben zodat het bloed efficiënt door het hele lichaam kan circuleren. De bloeddruk zorgt ervoor dat zuurstofrijk bloed de organen bereikt, zoals de hersenen en het hart, terwijl afvalstoffen worden afgevoerd. Te hoge bloeddruk kan leiden tot schade aan bloedvaten en organen, terwijl een te lage bloeddruk kan resulteren in onvoldoende zuurstoftoevoer naar belangrijke weefsels. Het is daarom cruciaal om de bloeddruk in balans te houden en afwijkingen tijdig te signaleren.

\subsubsection{Ademhaling}
Ademhaling (respiratory rate) is een van de meest fundamentele processen in het lichaam. Het zorgt ervoor dat zuurstof wordt opgenomen en koolstofdioxide wordt afgevoerd. Zuurstof is nodig voor de celademhaling, een proces waarbij energie wordt vrijgemaakt om lichaamsfuncties te ondersteunen. Een regelmatige ademhaling helpt bij het handhaven van de pH-balans in het bloed, wat essentieel is voor het functioneren van enzymen en andere biochemische processen. Veranderingen in de ademhalingsfrequentie kunnen wijzen op aandoeningen zoals longziekten, stress, of ademhalingsproblemen, en geven vaak een eerste indicatie van acute medische situaties.

\subsubsection{Lichaamstemperatuur}
Het lichaam moet op een bepaalde temperatuur (body temperature) zijn omdat veel biologische processen temperatuurgevoelig zijn. Enzymen, die verantwoordelijk zijn voor chemische reacties in het lichaam, werken optimaal binnen een smal temperatuurbereik. Een te lage lichaamstemperatuur kan leiden tot hypothermie, waarbij het lichaam niet meer in staat is om normaal te functioneren. Een te hoge temperatuur kan duiden op koorts of hyperthermie, wat vaak een reactie is op infecties of andere medische aandoeningen. Het handhaven van een stabiele lichaamstemperatuur is dus cruciaal voor de homeostase.

\subsubsection{Zuurstof}
Er moet voldoende zuurstof (oxygen saturation) in het lichaam zitten om de cellen van energie te voorzien. Zuurstof wordt via de longen opgenomen en door het bloed naar alle organen getransporteerd. Een tekort aan zuurstof kan leiden tot weefselbeschadiging, orgaanfalen, of zelfs levensbedreigende situaties. Het zuurstofniveau is een belangrijke indicator van hoe goed de longen en het hart samenwerken om het lichaam van zuurstof te voorzien. Door zuurstofverzadiging te monitoren, kunnen ernstige problemen zoals hypoxie vroegtijdig worden herkend.

\subsubsection{koolstofdioxide}
Er mag niet te veel koolstofdioxide (CO2 level) in het lichaam zijn omdat een overmaat kan leiden tot een verstoring van de pH-balans in het bloed, wat bekend staat als acidose. Dit kan gevaarlijk zijn voor de werking van cellen en organen. Koolstofdioxide wordt normaal gesproken afgevoerd via de ademhaling, en een ophoping kan wijzen op ademhalingsproblemen of aandoeningen zoals COPD. Het reguleren van het CO2-gehalte is daarom essentieel om het lichaam in een gezonde toestand te houden en symptomen zoals kortademigheid en vermoeidheid te voorkomen.


\subsection{Gegevens van biometrie}

\subsubsection{Heart beat rate}
De hartslag (heart beat rate) wordt gemeten in slagen per minuut (bpm). Een normaal bereik in rust ligt tussen de 60 en 100 bpm. Tijdens fysieke inspanning, stress, of emoties kan dit tijdelijk oplopen tot 150-180 bpm, afhankelijk van leeftijd en conditie. Een hartslag lager dan 60 bpm (bradycardie) kan normaal zijn bij getrainde atleten, maar kan ook wijzen op hartritmestoornissen of andere problemen, vooral als dit gepaard gaat met duizeligheid of vermoeidheid. Een hartslag hoger dan 100 bpm in rust (tachycardie) kan een teken zijn van koorts, stress, uitdroging, of hartproblemen. Kritieke waardes zijn een hartslag boven de 200 bpm of onder de 40 bpm, wat onmiddellijke medische aandacht vereist, omdat dit kan leiden tot onvoldoende doorbloeding van vitale organen.

\subsubsection{Blood pressure}
Bloeddruk (blood pressure) wordt gemeten in millimeter kwikdruk (mmHg). De systolische waarde (hoogste druk tijdens het samentrekken van het hart) en de diastolische waarde (druk tijdens de rustfase van het hart) worden weergegeven als een verhouding, bijvoorbeeld 120/80 mmHg. Een systolische druk onder 90 mmHg of een diastolische druk onder 60 mmHg wordt beschouwd als hypotensie (lage bloeddruk), wat kan leiden tot duizeligheid of flauwvallen. Een systolische druk boven 140 mmHg of een diastolische druk boven 90 mmHg wordt beschouwd als hypertensie (hoge bloeddruk) en kan het risico op hart- en vaatziekten vergroten. Kritieke waardes, zoals een systolische druk boven 180 mmHg of een diastolische druk boven 120 mmHg, vereisen onmiddellijke medische interventie.

\subsubsection{Respiratory rate}
De ademhalingsfrequentie (respiratory rate) wordt gemeten in aantal ademhalingen per minuut. Een normaal bereik in rust ligt tussen de 12 en 20 ademhalingen per minuut. Een frequentie onder de 12 ademhalingen per minuut (bradypneu) kan wijzen op ademhalingsproblemen, zoals slaapapneu of overdosis van bepaalde medicijnen. Een frequentie boven de 25 ademhalingen per minuut (tachypneu) kan een teken zijn van stress, koorts, of aandoeningen zoals longontsteking. Ademhalingsfrequenties lager dan 8 per minuut of hoger dan 30 per minuut worden als kritisch beschouwd, omdat dit kan leiden tot onvoldoende zuurstoftoevoer naar het lichaam of uitputting van het ademhalingssysteem.

\subsubsection{Body temperature}
De lichaamstemperatuur (body temperature) wordt gemeten in graden Celsius (°C). Een normale temperatuur varieert tussen de 36,5 en 37,5°C, afhankelijk van het moment van de dag en de meetmethode. Een temperatuur onder de 35°C wordt beschouwd als hypothermie, wat kan optreden door blootstelling aan koude omgevingen en kan leiden tot ernstige complicaties zoals orgaanfalen. Een temperatuur boven de 38°C duidt op koorts, meestal als gevolg van een infectie. Een lichaamstemperatuur boven de 41°C of onder de 32°C is levensbedreigend en vereist onmiddellijke medische hulp.

\subsubsection{Oxygen saturation}
Het zuurstofsaturatieniveau (oxygen saturation) wordt gemeten in procenten (\%), meestal met een pulsoximeter die het zuurstofgehalte in het bloed meet. Normale waarden liggen tussen de 95\% en 100\%. Een zuurstofniveau onder de 90\% wordt beschouwd als hypoxemie en kan wijzen op ademhalingsproblemen, longziekten, of een verminderde hartfunctie. Waarden onder de 85\% zijn kritisch, omdat dit kan leiden tot weefselschade, bewusteloosheid, of orgaanfalen.

\subsubsection{CO2 level}
Het koolstofdioxidegehalte (CO2 level) wordt meestal gemeten in millimeter kwikdruk (mmHg) in arterieel bloed. Normale waarden liggen tussen de 35 en 45 mmHg. Een CO2-gehalte onder de 35 mmHg duidt op hypocapnie, wat vaak het gevolg is van hyperventilatie en kan leiden tot symptomen zoals duizeligheid of tintelingen. Waarden boven de 45 mmHg wijzen op hypercapnie, wat kan worden veroorzaakt door aandoeningen zoals COPD of hypoventilatie. CO2-waarden boven de 60 mmHg zijn kritisch en kunnen respiratoire acidose veroorzaken, wat het zuur-base-evenwicht in het lichaam ernstig verstoort en onmiddellijke medische aandacht vereist.


\subsection{LM en CM}
De Lunar Module (LM) en de Command Module (CM) hebben allebei eigen levensondersteunende systemen. De LM is ontwikkeld voor twee personen om genoeg voorraad te hebben zodat als er een maanlanding plaats vindt, de astronauten genoeg hebben om terug naar het schip te reizen. 

Omdat de deur van de LM open kan gaan, is er veel meer zuurstof aanwezig dan eigenlijk nodig omdat de LM weer druk moet krijgen nadat de deur weer gesloten is.


\subsection{Mogelijke problemen}

\subsubsection{Drinkwater filter kapot}
Het kan zijn dat de filter voor het drinkwater kapot gaat. Dit zorgt ervoor dat de pH waarde zakt van 7 naar ongeveer 4. Ook ziet het water er bruin uit omdat het water ook als koeling fungeert voor de systemen aan boord. Hierdoor krijgt het water een ijzeren smaak.

Dit probleem is vrij makkelijk verholpen door de drinkwater filter aan te passen.

\subsubsection{Onzuivere lucht}
Wanneer er te veel onzuivere lucht in de capsule is, kan een luchtfilter gemaakt worden. Dit kan met spullen die gewoon in de raket liggen. In de bijlage is een idee over hoe dit kan gebeuren.

Wanneer er als gevolg van de slechte lucht te weinig zuurstof is, moet ervoor gezorgd worden dat de zuurstofsystemen en luchtfiltering weer gestart wordt, of de astronauten moeten hun best doen 
rustiger te ademen.

\subsubsection{Urineweginfectie}
Door slechte hygiëne is het mogelijk dat een astronaut een urineweginfectie krijgt. Hiervoor is Ciprofloxacine, een antibiotica, aanwezig aan boord. Een urineweginfectie is te herkennen aan pijn bij plassen, constant moeten plassen, en lichte koorts.


\subsection{Urgentie}
De urgentie van een probleem kan worden ingeschat door de biometrische gegevens zorgvuldig te interpreteren in de context van de situatie. Afwijkingen van de normale waarden kunnen een indicatie zijn van onderliggende problemen, maar de mate van urgentie hangt af van hoe ernstig de afwijking is, hoe snel deze zich ontwikkelt, en wat de mogelijke gevolgen zijn.

\subsubsection{Hoog}
Een hoge urgentie wordt toegekend wanneer een biometrisch gegeven zich in een kritisch bereik bevindt of wanneer snelle achteruitgang zichtbaar is. Voorbeelden zijn levensbedreigende situaties zoals een extreem lage hartslag, ernstige hypoxemie (zuurstofsaturatie <85\%), of een lichaamstemperatuur boven de 41°C of onder de 32°C. Deze situaties vereisen onmiddellijke medische interventie om schade aan vitale organen of de dood te voorkomen. Bij hoge urgentie moeten alle middelen worden ingezet om de toestand van de astronaut te stabiliseren.

\subsubsection{Middel}
Een middelmatige urgentie wordt toegekend wanneer een biometrisch gegeven buiten het normale bereik ligt, maar niet direct levensbedreigend is. Voorbeelden zijn een licht verhoogde hartslag (tachycardie), een lichte koorts (lichaamstemperatuur tussen 38°C en 39°C), of een ademhalingsfrequentie die iets te hoog of te laag is. Hoewel er geen onmiddellijke dreiging is, kan dit wijzen op een onderliggend probleem zoals uitdroging, stress, of een beginnende infectie. Het is belangrijk om deze situaties te monitoren en waar nodig maatregelen te treffen om verergering te voorkomen.

\subsubsection{Laag}
Een lage urgentie wordt toegekend wanneer de biometrische gegevens binnen een grensgebied vallen dat nog als acceptabel wordt beschouwd, of wanneer een afwijking te wijten is aan een verklaarbare en tijdelijke oorzaak, zoals fysieke inspanning of emotionele stress. In deze gevallen kan het voldoende zijn om de situatie te blijven observeren en eenvoudige preventieve maatregelen te nemen, zoals rust, hydratatie, of aanpassing van de omgevingstemperatuur. Zolang er geen verdere afwijkingen optreden, is er meestal geen interventie nodig.

