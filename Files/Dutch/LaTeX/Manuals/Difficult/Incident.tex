\section{Incident Manager}

Je bent aangesteld als Incident Manager binnen Mission Control, de ruggengraat van de Apollo 13-missie. Tijdens deze kritieke operatie ben jij verantwoordelijk voor het beheren van alle incidenten die zich voordoen. Denk aan een storing in een systeem, een probleem met de zuurstoftoevoer, of een onverwachte foutmelding die de missie in gevaar kan brengen. Jouw scherpzinnigheid en gestructureerde aanpak bepalen of de bemanning het avontuur overleeft en veilig terugkeert naar de aarde.

Als Incident Manager moet je snel en nauwkeurig werken, met beperkte informatie. Incidenten kunnen variëren in ernst en prioriteit, en het is aan jou om te beslissen welke problemen onmiddellijk aandacht vereisen en welke kunnen wachten. Je werkt nauw samen met de andere teamleden om oplossingen te vinden, terwijl je tegelijkertijd verslag uitbrengt aan de Teamleider.

\subsection{Verantwoordelijkheden}
Op elk moment van de missie ontvangt communicatie incidenten vanuit de raket. Deze incidenten worden opgeschreven op een 'incident report card' en naar jou gebracht.

Deze incident report card moet jij verder invullen. De volgende gegevens zijn aan jou om in te vullen. Natuurlijk kan je hierbij advies vragen aan de experts. Omdat je veel incidenten binnen zult krijgen, is er staff die het advies kan regelen voor de incidenten. De volgende gegevens moeten opgeschreven worden:
\begin{itemize}
    \item Urgentie (hoog, middel, of laag)
    \item Verantwoordelijke (welke expert het moet gaan oplossen)
    \item Probleem (wat het probleem is van de constatering)
    \item Actie (welke actie er ondernomen moet worden door de astronauten)
    \item Gecontroleerd door (nadat alles is ingevuld, schrijf je hier op wie het heeft gedaan)
\end{itemize}

Als alles is opgeschreven, kan de kaart terug naar communicatie die het daarna kan communiceren. wanneer dit is gebeurd, komt de kaart weer terug naar jou zodat sommige 'standaard' incidenten makkelijker en sneller opgelost kunnen worden.