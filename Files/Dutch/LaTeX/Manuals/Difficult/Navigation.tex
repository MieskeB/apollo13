\section{Navigatie expert}

Welkom in de rol van de Navigatie Expert. Jouw primaire missiedoel is ervoor zorgen dat de astronauten veilig terugkeren naar de aarde en dat de missie zo nauwkeurig mogelijk de vooraf gestelde doelen bereikt. Dit betekent dat je voortdurend de koers moet controleren, bijsturen waar nodig, en de sterrenkaart moet gebruiken om de oriëntatie van het ruimtevaartuig te bepalen. Navigatie in de ruimte vereist precisie, snel handelen, en een goed begrip van zowel de technische hulpmiddelen als de astronomische gegevens. Deze uitgebreide handleiding voorziet je van de nodige kennis en procedures om deze cruciale rol te vervullen.

\subsection{Overzicht van sterren}
Sterren spelen een cruciale rol in de navigatie van een ruimtevaartuig. Omdat sterren vaste posities hebben ten opzichte van elkaar, dienen ze als betrouwbare referentiepunten in een anders veranderlijk en onbekend ruimtelandschap. De sterrenkaart bevat gedetailleerde informatie over de meest zichtbare sterren vanaf de positie van het ruimteschip, rekening houdend met factoren zoals lichtvervuiling van de aarde en de zon.

De navigatie-instrumenten maken gebruik van een sextant, waarmee de hoeken tussen specifieke sterren en de horizon van de aarde of maan konden worden gemeten. Veelgebruikte sterren tijdens de missie zijn Sirius, Canopus, en Regulus, omdat deze helder en gemakkelijk te herkennen zijn. Deze sterren werden vergeleken met hun berekende posities op een vooraf geladen sterrenkaart om de huidige locatie en richting van het ruimteschip te bepalen.

Daarnaast veranderen de relatieve posities van sterren gedurende de reis door de ruimte, wat bekend staat als stellaire parallax. Hoewel deze veranderingen meestal klein zijn, moesten ze nauwkeurig worden gecorrigeerd om koersberekeningen te verfijnen. Navigeren met sterren vereist een combinatie van technologische precisie, astronomische kennis, en menselijke intuïtie.


\subsection{Gegevens van koers}
De koers kan worden bepaald door een combinatie van parameters: positie, snelheid, en oriëntatie. Dit traject wordt beïnvloed door een complex samenspel van krachten, zoals de zwaartekracht van de aarde, de maan, en de zon, evenals de impuls van de motoren en de weerstand van zonnedeeltjes. Een nauwkeurige berekening van de koers was essentieel om te zorgen dat het ruimteschip op de juiste manier om de maan slingerde en terugkeerde naar de aarde.

\subsubsection{Positiebepaling}
De positie wordt bepaald door triangulatie met behulp van de Deep Space Network (DSN), een netwerk van gigantische antennes op aarde. Deze metingen geven een exacte locatie van het ruimteschip ten opzichte van aarde en maan. Tegelijkertijd helpen stermetingen de bemanning om de oriëntatie van het ruimteschip te bevestigen.

Wanneer DSN niet beschikbaar is, moet er overgegaan worden op sterrenlezen met de bijgevoegde sterrenkaart.

\subsubsection{Snelheid}
De snelheid van het ruimteschip wordt voortdurend bijgehouden en aangepast om ervoor te zorgen dat het binnen de vereiste parameters blijft.

\subsubsection{Oriëntatie en attitudecontrole}
De oriëntatie van het ruimteschip, oftewel attitude, wordt gecontroleerd door middel van gyroscopen en kleine stuwraketten, bekend als Reaction Control System (RCS). Deze systemen houden het ruimteschip in de juiste hoek ten opzichte van de aarde, maan, en zon, om te zorgen voor efficiënte communicatie en energieopwekking via de zonnepanelen.


\subsection{Mogelijke problemen}

\subsubsection{Te weinig stuwkracht}
Wanneer het zichtbaar is dat er te weinig fuel is, is het mogelijk om bijzondere verrichtingen uit te voeren om hierop te besparen. Deze verrichtingen staan beschreven in het gelijknamige hoofdstuk hieronder.

\subsubsection{Navigatiesysteem kapot}
In het geval dat een navigatiesysteem kapot is, probeer hem dan altijd als eerste te herstarten. De navigatieapparatuur werkt accurater dan het oog. Mocht dit niet lukken, moet er met het oog genavigeerd worden. De astronauten hebben de sterrenkaart van buiten geleerd en kunnen vanuit de raket precies aanwijzen welke sterren waar zijn. Op basis hiervan, samen met de sextant die gebruikt wordt, kunnen de astronauten met de juiste sterrennamen en de juiste hoeken precies hun koers bepalen.


\subsection{Bijzondere verrichtingen}

\subsubsection{Zwaartekrachtslinger}
De zwaartekrachtslinger gebruikt de zwaartekracht van een hemellichaam om snelheid en richting aan te passen zonder extra brandstof te gebruiken.

Uitvoering:
\begin{enumerate}
    \item Plan de koers nauwkeurig om een groot hemellichaam te passeren.
    \item Gebruik de orbit van het hemellichaam om makkelijk van richting te veranderen.
    \item Wanneer de raket juist georiënteerd is, stuw de raket uit de orbit en vervolg de reis.
\end{enumerate}

\subsubsection{Plotseling omdraaien}
Plotseling omdraaien is mogelijk, maar kost heel veel brandstof.

Uitvoering:
\begin{enumerate}
    \item Voer een volledige rotatie uit met behulp van de RCS.
    \item Gebruik de hoofdmotor om de raket te vertragen en daarna te versnellen in de andere richting.
\end{enumerate}

\subsubsection{Zonnezeilspanning}
Door de zonnepanelen op de zon te richten, kan er extra energie opgewekt worden waardoor er extra voortstuwing gecreëerd kan worden.

Uitvoering:
\begin{enumerate}
    \item Roteer de raket zodat de zonnepanelen op de zon gericht staan.
    \item Probeer het ruimteschip langzaam extra voort te stuwen.
\end{enumerate}

\subsubsection{Vrije val}
De motoren kunnen uitgeschakeld worden en 'vrij' door de ruimte vallen in de hoop dat natuurlijke krachten het schip naar de juiste koers brengen.

Uitvoering:
\begin{enumerate}
    \item Stop alle motoren en vertrouw volledig op het bestaande momentum.
    \item Wacht op het effect van externe krachten en hoop dat alles goed komt.
\end{enumerate}


\subsection{Urgentie}
De urgentie van een navigatieprobleem wordt bepaald door de omvang van de koersafwijking en de gevolgen daarvan voor de missie.

\subsubsection{Hoog}
Een probleem wordt als urgent beschouwd wanneer de koers ernstig afwijkt en de veiligheid van de bemanning of het bereiken van de bestemming direct in gevaar komt. Voorbeelden hiervan zijn een dreigende botsing met een hemellichaam of het missen van de aarde. In dergelijke gevallen moet je onmiddellijk handelen om het traject aan te passen, zelfs als dit andere missieprioriteiten beïnvloedt.

\subsubsection{Middel}
Middelmatige urgentie treedt op wanneer er een significante koersafwijking is die nog niet kritisch is, maar die bij uitblijven van correcties de missie kan beïnvloeden. Dit kan bijvoorbeeld het gevolg zijn van een kleine fout in de navigatiegegevens of een lichte verstoring van de motoren. Deze situaties vereisen snelle aandacht, maar je hebt doorgaans meer tijd om een oplossing te bedenken en uit te voeren.

\subsubsection{Laag}
Problemen met lage urgentie omvatten kleine afwijkingen die geen onmiddellijke impact hebben op de missie en die kunnen worden opgelost tijdens routinecontroles. Bijvoorbeeld, een lichte afwijking in de positie van een ster op de kaart of een geringe afwijking in snelheid. Deze problemen kunnen worden aangepakt zonder de normale missie-operaties te onderbreken.
